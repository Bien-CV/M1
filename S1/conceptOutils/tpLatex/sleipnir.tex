\documentclass[a4paper, 12pt]{article}
\title{SLEIPNIR, l'octopode fabuleux}
\usepackage[latin1]{inputenc}
\usepackage[T1]{fontenc}
\usepackage{hyperref}
\usepackage[francais]{babel}
\usepackage{multicol}
\usepackage{dialogue}
\usepackage{floatflt}


\hypersetup{colorlinks=false}


\begin{document}
\date{\href{http://fr.wikipedia.org}{fr.wikipedia.org} }
\maketitle
\begin{abstract}

Dans la mythologie scandinave, Sleipnir est le cheval fabuleux à
  huit jambes d'Odin. On présente ici succinctement l'origine de ce
  mythe et sa représentation.
\end{abstract}
\section{Introduction}

Sleipnir est, dans la mythologie nordique, un cheval fabuleux à huit
jambes capable de se déplacer au-dessus de la mer comme dans les airs,
monture habituelle du dieu Odin.  Il est mentionné dans
l?Edda poétique ~\cite{Livre} , série de textes compilés au XIII
siècle à partir de sources plus anciennes, et dans l?Edda en prose,
rédigée à la même époque par Snorri Sturluson. Selon ces deux sources,
Sleipnir est le fils du dieu Loki et d'un puissant étalon,
Svaðilfari. Décrit comme « le meilleur de tous les chevaux » et le
plus rapide, il devient la monture d'Odin qui le chevauche jusque dans
la région de Hel ; toutefois, le dieu s'en sert surtout pour traverser
le pont Bifröst afin de se rendre à la troisième racine
d'Yggdrasil , là où se tient le conseil des
dieux. L'Edda en prose donne de nombreux détails sur les circonstances
de la naissance de Sleipnir et précise, par exemple, qu'il est de
couleur grise.

Sleipnir est également mentionné dans une énigme figurant dans une
saga légendaire du XIII siècle, la Saga de Hervor et du roi
Heidrekr, ainsi que dans la Völsunga saga, comme ancêtre du cheval
Grani. L'un des livres de la geste des Danois de Saxo Grammaticus au
XIII siècle contient aussi un épisode qui, selon de nombreux
érudits, concernerait ce cheval. Il est généralement admit que
Sleipnir fut représenté sur plusieurs pierres historiées de Gotland
vers le VIII siècle, notamment la pierre de Tjängvide et la
pierre d'Ardre VIII.

De nombreuses théories ont été proposées pour décrypter la symbolique
de Sleipnir et sa possible relation avec des pratiques chamaniques à
l'époque du paganisme nordique.

Selon le folklore islandais, Sleipnir est aussi le créateur du canyon
d'Ásbyrgi. À l'époque moderne, son mythe et sa symbolique sont
abondamment repris dans l'art et la littérature ; ainsi, il a
probablement inspiré Tolkien pour créer le cheval Gripoil (en anglais,
Shadowfax), monture de Gandalf, et son nom a été donné,
entre autres, à plusieurs navires ainsi qu'à un navigateur web.

\section{Mentions dans les anciens textes}
On procédera en plusieurs étapes: Les Eddas fournissent de nombreux
renseignements sur ce cheval, qui possède pour caractéristiques
constantes le fait d'avoir huit jambes, et d'être décrit comme « le
meilleur de tous les chevaux ».

\subsection{Edda en prose}
Dans l'un des livres de l?Edda en prose (Figure ),
Gylfaginning, Sleipnir est mentionné pour la première fois au chapitre
15 quand Hár raconte que chaque jour, les Ases chevauchent à travers
le pont Bifröst, puis donne la liste de leurs chevaux. Cette liste
commence avec Sleipnir : « Le meilleur d'entre eux est Sleipnir, il
appartient à Odin et a huit jambes ». Au chapitre 41, Hár cite une
strophe qui mentionne Sleipnir dans le Grímnismál.




\subsection{Saga de Hervor et du roi Heidrekr}
Dans la Saga de Hervor et du roi Heidrekr (Hervarar saga ok Heiðreks),
le poème Heiðreks gátur contient une énigme qui mentionne Sleipnir et
Odin :
\begin{multicols}{2}  
  
	\paragraph{Texte original, 36.}
	\begin{dialogue}
	\speak{Gestumblindi}

    \begin{verse}
    Hverir eru þeir tveir, 
    er tíu hafa fætr, 
    augu þrjú ok einn hala? 
    Heiðrekr konungr, 
    hyggðuat gátu
     \end{verse}
 
    \speak{Heiðreks}

    \begin{verse}
    Góð er gáta þín, Gestumblindi, Þat er þá, er Óðinn ríðr Sleipni.
    \end{verse}
  
  \end{dialogue}




  \paragraph{Traduction française, 36.}
  Gestumblindi :

      Qui sont les deux

      qui courent, sur dix pieds,

      trois yeux ils ont,

      mais une seule queue ?

      Allez, réponds maintenant

      à cette énigme, Heidrek.

  Heidrek :

      Ton énigme est bonne, Gestumblindi,

      et je l'ai trouvée, c'est Odin qui chevauche Sleipnir.
\end{multicols}
\tableofcontents

\bibliographystyle{plain}
\bibliography{biblio}
\end{document}
